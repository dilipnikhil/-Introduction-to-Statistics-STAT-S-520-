% Options for packages loaded elsewhere
\PassOptionsToPackage{unicode}{hyperref}
\PassOptionsToPackage{hyphens}{url}
%
\documentclass[
]{article}
\usepackage{amsmath,amssymb}
\usepackage{iftex}
\ifPDFTeX
  \usepackage[T1]{fontenc}
  \usepackage[utf8]{inputenc}
  \usepackage{textcomp} % provide euro and other symbols
\else % if luatex or xetex
  \usepackage{unicode-math} % this also loads fontspec
  \defaultfontfeatures{Scale=MatchLowercase}
  \defaultfontfeatures[\rmfamily]{Ligatures=TeX,Scale=1}
\fi
\usepackage{lmodern}
\ifPDFTeX\else
  % xetex/luatex font selection
\fi
% Use upquote if available, for straight quotes in verbatim environments
\IfFileExists{upquote.sty}{\usepackage{upquote}}{}
\IfFileExists{microtype.sty}{% use microtype if available
  \usepackage[]{microtype}
  \UseMicrotypeSet[protrusion]{basicmath} % disable protrusion for tt fonts
}{}
\makeatletter
\@ifundefined{KOMAClassName}{% if non-KOMA class
  \IfFileExists{parskip.sty}{%
    \usepackage{parskip}
  }{% else
    \setlength{\parindent}{0pt}
    \setlength{\parskip}{6pt plus 2pt minus 1pt}}
}{% if KOMA class
  \KOMAoptions{parskip=half}}
\makeatother
\usepackage{xcolor}
\usepackage[margin=1in]{geometry}
\usepackage{color}
\usepackage{fancyvrb}
\newcommand{\VerbBar}{|}
\newcommand{\VERB}{\Verb[commandchars=\\\{\}]}
\DefineVerbatimEnvironment{Highlighting}{Verbatim}{commandchars=\\\{\}}
% Add ',fontsize=\small' for more characters per line
\usepackage{framed}
\definecolor{shadecolor}{RGB}{248,248,248}
\newenvironment{Shaded}{\begin{snugshade}}{\end{snugshade}}
\newcommand{\AlertTok}[1]{\textcolor[rgb]{0.94,0.16,0.16}{#1}}
\newcommand{\AnnotationTok}[1]{\textcolor[rgb]{0.56,0.35,0.01}{\textbf{\textit{#1}}}}
\newcommand{\AttributeTok}[1]{\textcolor[rgb]{0.13,0.29,0.53}{#1}}
\newcommand{\BaseNTok}[1]{\textcolor[rgb]{0.00,0.00,0.81}{#1}}
\newcommand{\BuiltInTok}[1]{#1}
\newcommand{\CharTok}[1]{\textcolor[rgb]{0.31,0.60,0.02}{#1}}
\newcommand{\CommentTok}[1]{\textcolor[rgb]{0.56,0.35,0.01}{\textit{#1}}}
\newcommand{\CommentVarTok}[1]{\textcolor[rgb]{0.56,0.35,0.01}{\textbf{\textit{#1}}}}
\newcommand{\ConstantTok}[1]{\textcolor[rgb]{0.56,0.35,0.01}{#1}}
\newcommand{\ControlFlowTok}[1]{\textcolor[rgb]{0.13,0.29,0.53}{\textbf{#1}}}
\newcommand{\DataTypeTok}[1]{\textcolor[rgb]{0.13,0.29,0.53}{#1}}
\newcommand{\DecValTok}[1]{\textcolor[rgb]{0.00,0.00,0.81}{#1}}
\newcommand{\DocumentationTok}[1]{\textcolor[rgb]{0.56,0.35,0.01}{\textbf{\textit{#1}}}}
\newcommand{\ErrorTok}[1]{\textcolor[rgb]{0.64,0.00,0.00}{\textbf{#1}}}
\newcommand{\ExtensionTok}[1]{#1}
\newcommand{\FloatTok}[1]{\textcolor[rgb]{0.00,0.00,0.81}{#1}}
\newcommand{\FunctionTok}[1]{\textcolor[rgb]{0.13,0.29,0.53}{\textbf{#1}}}
\newcommand{\ImportTok}[1]{#1}
\newcommand{\InformationTok}[1]{\textcolor[rgb]{0.56,0.35,0.01}{\textbf{\textit{#1}}}}
\newcommand{\KeywordTok}[1]{\textcolor[rgb]{0.13,0.29,0.53}{\textbf{#1}}}
\newcommand{\NormalTok}[1]{#1}
\newcommand{\OperatorTok}[1]{\textcolor[rgb]{0.81,0.36,0.00}{\textbf{#1}}}
\newcommand{\OtherTok}[1]{\textcolor[rgb]{0.56,0.35,0.01}{#1}}
\newcommand{\PreprocessorTok}[1]{\textcolor[rgb]{0.56,0.35,0.01}{\textit{#1}}}
\newcommand{\RegionMarkerTok}[1]{#1}
\newcommand{\SpecialCharTok}[1]{\textcolor[rgb]{0.81,0.36,0.00}{\textbf{#1}}}
\newcommand{\SpecialStringTok}[1]{\textcolor[rgb]{0.31,0.60,0.02}{#1}}
\newcommand{\StringTok}[1]{\textcolor[rgb]{0.31,0.60,0.02}{#1}}
\newcommand{\VariableTok}[1]{\textcolor[rgb]{0.00,0.00,0.00}{#1}}
\newcommand{\VerbatimStringTok}[1]{\textcolor[rgb]{0.31,0.60,0.02}{#1}}
\newcommand{\WarningTok}[1]{\textcolor[rgb]{0.56,0.35,0.01}{\textbf{\textit{#1}}}}
\usepackage{graphicx}
\makeatletter
\def\maxwidth{\ifdim\Gin@nat@width>\linewidth\linewidth\else\Gin@nat@width\fi}
\def\maxheight{\ifdim\Gin@nat@height>\textheight\textheight\else\Gin@nat@height\fi}
\makeatother
% Scale images if necessary, so that they will not overflow the page
% margins by default, and it is still possible to overwrite the defaults
% using explicit options in \includegraphics[width, height, ...]{}
\setkeys{Gin}{width=\maxwidth,height=\maxheight,keepaspectratio}
% Set default figure placement to htbp
\makeatletter
\def\fps@figure{htbp}
\makeatother
\setlength{\emergencystretch}{3em} % prevent overfull lines
\providecommand{\tightlist}{%
  \setlength{\itemsep}{0pt}\setlength{\parskip}{0pt}}
\setcounter{secnumdepth}{-\maxdimen} % remove section numbering
\ifLuaTeX
  \usepackage{selnolig}  % disable illegal ligatures
\fi
\IfFileExists{bookmark.sty}{\usepackage{bookmark}}{\usepackage{hyperref}}
\IfFileExists{xurl.sty}{\usepackage{xurl}}{} % add URL line breaks if available
\urlstyle{same}
\hypersetup{
  pdftitle={PS12},
  pdfauthor={Dilip Nikhil Francies},
  hidelinks,
  pdfcreator={LaTeX via pandoc}}

\title{PS12}
\author{Dilip Nikhil Francies}
\date{2023-12-07}

\begin{document}
\maketitle

\begin{Shaded}
\begin{Highlighting}[]
\FunctionTok{getwd}\NormalTok{()}
\end{Highlighting}
\end{Shaded}

\begin{verbatim}
## [1] "C:/Users/dilip/OneDrive - Indiana University/Stats/PS12"
\end{verbatim}

\hypertarget{question-1-brother-vs-sister}{%
\subsubsection{Question 1: Brother vs
Sister}\label{question-1-brother-vs-sister}}

Lets read the data as vectors:

\begin{Shaded}
\begin{Highlighting}[]
\NormalTok{sister }\OtherTok{\textless{}{-}} \FunctionTok{c}\NormalTok{(}\DecValTok{69}\NormalTok{, }\DecValTok{64}\NormalTok{, }\DecValTok{65}\NormalTok{, }\DecValTok{63}\NormalTok{, }\DecValTok{65}\NormalTok{, }\DecValTok{62}\NormalTok{, }\DecValTok{65}\NormalTok{, }\DecValTok{64}\NormalTok{, }\DecValTok{66}\NormalTok{, }\DecValTok{59}\NormalTok{, }\DecValTok{62}\NormalTok{)}
\NormalTok{brother }\OtherTok{\textless{}{-}} \FunctionTok{c}\NormalTok{(}\DecValTok{71}\NormalTok{, }\DecValTok{68}\NormalTok{, }\DecValTok{66}\NormalTok{, }\DecValTok{67}\NormalTok{, }\DecValTok{70}\NormalTok{, }\DecValTok{71}\NormalTok{, }\DecValTok{70}\NormalTok{, }\DecValTok{73}\NormalTok{, }\DecValTok{72}\NormalTok{, }\DecValTok{65}\NormalTok{, }\DecValTok{66}\NormalTok{)}
\end{Highlighting}
\end{Shaded}

\hypertarget{question-1a-sample-coeffecient-of-determination-rsquared}{%
\subsubsection{Question 1a: Sample Coeffecient of Determination :
Rsquared}\label{question-1a-sample-coeffecient-of-determination-rsquared}}

\begin{Shaded}
\begin{Highlighting}[]
\NormalTok{lin\_model }\OtherTok{\textless{}{-}} \FunctionTok{lm}\NormalTok{(brother }\SpecialCharTok{\textasciitilde{}}\NormalTok{ sister)}
\NormalTok{rSquared }\OtherTok{\textless{}{-}} \FunctionTok{summary}\NormalTok{(lin\_model)}\SpecialCharTok{$}\NormalTok{r.squared}
\NormalTok{rSquared}
\end{Highlighting}
\end{Shaded}

\begin{verbatim}
## [1] 0.3114251
\end{verbatim}

\hypertarget{question-1b}{%
\subsubsection{Question 1b:}\label{question-1b}}

alpha = 0.05

Lets find the p-value for the hypothesis that the slope in the linear
regression model is 0, and the alternate hypothesis that it is not.

\begin{Shaded}
\begin{Highlighting}[]
\FunctionTok{summary}\NormalTok{(lin\_model)}
\end{Highlighting}
\end{Shaded}

\begin{verbatim}
## 
## Call:
## lm(formula = brother ~ sister)
## 
## Residuals:
##     Min      1Q  Median      3Q     Max 
## -3.5909 -1.2273 -0.9545  1.1136  4.0000 
## 
## Coefficients:
##             Estimate Std. Error t value Pr(>|t|)  
## (Intercept)  31.1818    18.7584   1.662   0.1308  
## sister        0.5909     0.2929   2.018   0.0744 .
## ---
## Signif. codes:  0 '***' 0.001 '**' 0.01 '*' 0.05 '.' 0.1 ' ' 1
## 
## Residual standard error: 2.379 on 9 degrees of freedom
## Multiple R-squared:  0.3114, Adjusted R-squared:  0.2349 
## F-statistic:  4.07 on 1 and 9 DF,  p-value: 0.07442
\end{verbatim}

With the p-value of around 0.07442, we can safely conclude that the data
is compatible with the null hypothesis that knowing sisters height is
not enough to predict brothers height.

\hypertarget{question-1c}{%
\subsubsection{Question 1c:}\label{question-1c}}

\begin{Shaded}
\begin{Highlighting}[]
\CommentTok{\# Confidence interval for the slope}
\NormalTok{conf\_interval }\OtherTok{\textless{}{-}} \FunctionTok{confint}\NormalTok{(lin\_model, }\AttributeTok{level =} \FloatTok{0.90}\NormalTok{)}
\NormalTok{conf\_interval}
\end{Highlighting}
\end{Shaded}

\begin{verbatim}
##                     5 %      95 %
## (Intercept) -3.20446954 65.568106
## sister       0.05401643  1.127802
\end{verbatim}

The 90\% confidence interval for the slope is 0.054 to 1.127.

\hypertarget{question-2-anxiety-and-exams}{%
\subsubsection{Question 2: Anxiety and
Exams}\label{question-2-anxiety-and-exams}}

Read the given data:

\begin{Shaded}
\begin{Highlighting}[]
\NormalTok{exam\_anxiety }\OtherTok{\textless{}{-}} \FunctionTok{read.csv}\NormalTok{(}\StringTok{"examanxiety.txt"}\NormalTok{,}\AttributeTok{sep =} \StringTok{"}\SpecialCharTok{\textbackslash{}t}\StringTok{"}\NormalTok{)}
\FunctionTok{head}\NormalTok{(exam\_anxiety,}\DecValTok{2}\NormalTok{)}
\end{Highlighting}
\end{Shaded}

\begin{verbatim}
##   Code Revise Exam Anxiety Gender
## 1    1      4   40  86.298   Male
## 2    2     11   65  88.716 Female
\end{verbatim}

Split the data:

\begin{Shaded}
\begin{Highlighting}[]
\NormalTok{anxiety\_male }\OtherTok{\textless{}{-}}\NormalTok{ exam\_anxiety}\SpecialCharTok{$}\NormalTok{Anxiety[exam\_anxiety}\SpecialCharTok{$}\NormalTok{Gender }\SpecialCharTok{==}\StringTok{"Male"}\NormalTok{]}
\NormalTok{anxiety\_feamale }\OtherTok{\textless{}{-}}\NormalTok{ exam\_anxiety}\SpecialCharTok{$}\NormalTok{Anxiety[exam\_anxiety}\SpecialCharTok{$}\NormalTok{Gender }\SpecialCharTok{==}\StringTok{"Female"}\NormalTok{]}
\end{Highlighting}
\end{Shaded}

\hypertarget{question-2a}{%
\paragraph{Question 2a:}\label{question-2a}}

Lets perform two sample test with the following hypothesis:

Null : There is no mean difference in anxiety levels between males and
females ie. muFemales = muMales Alternate hypotheses: There is
difference betweeen anxiety levels between males and females ie
muFemales != muMales

\begin{Shaded}
\begin{Highlighting}[]
\NormalTok{result }\OtherTok{\textless{}{-}} \FunctionTok{t.test}\NormalTok{(anxiety\_male,anxiety\_feamale,}\AttributeTok{var.equal =}\ConstantTok{FALSE}\NormalTok{)}
\FunctionTok{print}\NormalTok{(result)}
\end{Highlighting}
\end{Shaded}

\begin{verbatim}
## 
##  Welch Two Sample t-test
## 
## data:  anxiety_male and anxiety_feamale
## t = -0.32961, df = 100.41, p-value = 0.7424
## alternative hypothesis: true difference in means is not equal to 0
## 95 percent confidence interval:
##  -7.147444  5.110827
## sample estimates:
## mean of x mean of y 
##  74.38373  75.40204
\end{verbatim}

From the above results, its clear that there is no difference as the
p-value obtained is not tiny.

\hypertarget{question-2b}{%
\paragraph{Question 2b:}\label{question-2b}}

Lets draw the scatter plot with the regression line:

\begin{Shaded}
\begin{Highlighting}[]
\FunctionTok{library}\NormalTok{(ggplot2)}
\end{Highlighting}
\end{Shaded}

\begin{verbatim}
## Warning: package 'ggplot2' was built under R version 4.2.3
\end{verbatim}

\begin{Shaded}
\begin{Highlighting}[]
\NormalTok{scatterplot }\OtherTok{\textless{}{-}} \FunctionTok{ggplot}\NormalTok{(exam\_anxiety, }\FunctionTok{aes}\NormalTok{(}\AttributeTok{x =}\NormalTok{ Anxiety, }\AttributeTok{y =}\NormalTok{ Exam)) }\SpecialCharTok{+}
  \FunctionTok{geom\_point}\NormalTok{() }\SpecialCharTok{+}  \CommentTok{\# Scatterplot}
  \FunctionTok{geom\_smooth}\NormalTok{(}\AttributeTok{method =} \StringTok{"lm"}\NormalTok{, }\AttributeTok{se =} \ConstantTok{FALSE}\NormalTok{, }\AttributeTok{color =} \StringTok{"blue"}\NormalTok{) }\SpecialCharTok{+}  \CommentTok{\# Add regression line}
  \FunctionTok{labs}\NormalTok{(}\AttributeTok{x =} \StringTok{"Anxiety"}\NormalTok{, }\AttributeTok{y =} \StringTok{"Exam score"}\NormalTok{, }\AttributeTok{title =} \StringTok{"Scatterplot of Anxiety vs Exam score with Least Squares Regression Line"}\NormalTok{)}

\CommentTok{\# Print the plot}
\FunctionTok{print}\NormalTok{(scatterplot)}
\end{Highlighting}
\end{Shaded}

\begin{verbatim}
## `geom_smooth()` using formula = 'y ~ x'
\end{verbatim}

\includegraphics{PS12-Dilip_files/figure-latex/unnamed-chunk-9-1.pdf}

\begin{Shaded}
\begin{Highlighting}[]
\NormalTok{model }\OtherTok{\textless{}{-}} \FunctionTok{lm}\NormalTok{(Exam }\SpecialCharTok{\textasciitilde{}}\NormalTok{ Anxiety, }\AttributeTok{data =}\NormalTok{ exam\_anxiety)}
\FunctionTok{summary}\NormalTok{(model)}
\end{Highlighting}
\end{Shaded}

\begin{verbatim}
## 
## Call:
## lm(formula = Exam ~ Anxiety, data = exam_anxiety)
## 
## Residuals:
##     Min      1Q  Median      3Q     Max 
## -49.185 -16.046   1.166  19.856  41.461 
## 
## Coefficients:
##             Estimate Std. Error t value Pr(>|t|)    
## (Intercept) 111.2444    11.3498   9.801 2.46e-16 ***
## Anxiety      -0.7300     0.1484  -4.920 3.37e-06 ***
## ---
## Signif. codes:  0 '***' 0.001 '**' 0.01 '*' 0.05 '.' 0.1 ' ' 1
## 
## Residual standard error: 23.41 on 101 degrees of freedom
## Multiple R-squared:  0.1933, Adjusted R-squared:  0.1853 
## F-statistic:  24.2 on 1 and 101 DF,  p-value: 3.374e-06
\end{verbatim}

From the results and the plot,it looks like there is a downward trend in
the data, When the anxiety levels are higher, the marks tend to be all
over the place.The regression model does a horrible job in in capturing
the variance in the data set.Linear Regression may not be our best model
in predicting the exam score based on anxiety levels.

\begin{Shaded}
\begin{Highlighting}[]
\FunctionTok{library}\NormalTok{(broom)}
\end{Highlighting}
\end{Shaded}

\begin{verbatim}
## Warning: package 'broom' was built under R version 4.2.3
\end{verbatim}

\hypertarget{question-2c}{%
\paragraph{Question 2c:}\label{question-2c}}

\begin{enumerate}
\def\labelenumi{\roman{enumi}.}
\tightlist
\item
  Linearity:
\end{enumerate}

Scatter plot:

\begin{Shaded}
\begin{Highlighting}[]
\NormalTok{my.lm.df }\OtherTok{\textless{}{-}} \FunctionTok{augment}\NormalTok{(model)}
\FunctionTok{ggplot}\NormalTok{(my.lm.df, }\FunctionTok{aes}\NormalTok{(exam\_anxiety}\SpecialCharTok{$}\NormalTok{Anxiety, .resid)) }\SpecialCharTok{+} \FunctionTok{geom\_point}\NormalTok{() }\SpecialCharTok{+} \FunctionTok{geom\_smooth}\NormalTok{()}
\end{Highlighting}
\end{Shaded}

\begin{verbatim}
## `geom_smooth()` using method = 'loess' and formula = 'y ~ x'
\end{verbatim}

\includegraphics{PS12-Dilip_files/figure-latex/unnamed-chunk-12-1.pdf}

It is clear that there is no strong linear relationship between the two
variables. There does seem to be a lot of outliers, and most of the data
seems to be scattered on the right side of the plot with varying exam
scores for a small range of anxiety levels.

ii.Independence.

Let's plot a residuals vs.~fitted values plot.

\begin{Shaded}
\begin{Highlighting}[]
\NormalTok{model }\OtherTok{\textless{}{-}} \FunctionTok{lm}\NormalTok{(Exam }\SpecialCharTok{\textasciitilde{}}\NormalTok{ Anxiety, }\AttributeTok{data =}\NormalTok{ exam\_anxiety)}
\NormalTok{fit }\OtherTok{\textless{}{-}} \FunctionTok{fitted.values}\NormalTok{(model)}
\NormalTok{res }\OtherTok{\textless{}{-}} \FunctionTok{residuals}\NormalTok{(model)}
\FunctionTok{plot}\NormalTok{(fit, res, }\AttributeTok{main =} \StringTok{"Residuals vs Fitted"}\NormalTok{, }\AttributeTok{xlab =} \StringTok{"Fitted values"}\NormalTok{, }\AttributeTok{ylab =} \StringTok{"Residuals"}\NormalTok{)}
\end{Highlighting}
\end{Shaded}

\includegraphics{PS12-Dilip_files/figure-latex/unnamed-chunk-13-1.pdf}

Because the exam scores recorded are for different students, ideally the
data set should be independent as long as there were no multiple
measurements taken for the same students.

\begin{enumerate}
\def\labelenumi{\roman{enumi}.}
\setcounter{enumi}{2}
\tightlist
\item
  Homoskedasticity:
\end{enumerate}

\begin{Shaded}
\begin{Highlighting}[]
\FunctionTok{ggplot}\NormalTok{(model, }\FunctionTok{aes}\NormalTok{(}\AttributeTok{x =} \FunctionTok{fitted}\NormalTok{(model), }\AttributeTok{y =} \FunctionTok{resid}\NormalTok{(model))) }\SpecialCharTok{+}
  \FunctionTok{geom\_point}\NormalTok{() }\SpecialCharTok{+}
  \FunctionTok{geom\_hline}\NormalTok{(}\AttributeTok{yintercept =} \DecValTok{0}\NormalTok{, }\AttributeTok{linetype =} \StringTok{"dashed"}\NormalTok{, }\AttributeTok{color =} \StringTok{"red"}\NormalTok{) }\SpecialCharTok{+}
  \FunctionTok{labs}\NormalTok{(}\AttributeTok{title =} \StringTok{"Residuals vs Fitted Values"}\NormalTok{,}
       \AttributeTok{x =} \StringTok{"Fitted Values"}\NormalTok{,}
       \AttributeTok{y =} \StringTok{"Residuals"}\NormalTok{)}
\end{Highlighting}
\end{Shaded}

\includegraphics{PS12-Dilip_files/figure-latex/unnamed-chunk-14-1.pdf}

Let's plot absolute residual plot:

\begin{Shaded}
\begin{Highlighting}[]
\NormalTok{aug\_model\_df }\OtherTok{\textless{}{-}} \FunctionTok{augment}\NormalTok{(model)}
\FunctionTok{ggplot}\NormalTok{(aug\_model\_df, }\FunctionTok{aes}\NormalTok{(exam\_anxiety}\SpecialCharTok{$}\NormalTok{Anxiety, .resid)) }\SpecialCharTok{+} \FunctionTok{geom\_point}\NormalTok{() }\SpecialCharTok{+} \FunctionTok{geom\_smooth}\NormalTok{()}
\end{Highlighting}
\end{Shaded}

\begin{verbatim}
## `geom_smooth()` using method = 'loess' and formula = 'y ~ x'
\end{verbatim}

\includegraphics{PS12-Dilip_files/figure-latex/unnamed-chunk-15-1.pdf}
The residuals get smaller as we go right, and hence there is no constant
spread. Homoskedasticity check fails.

\begin{Shaded}
\begin{Highlighting}[]
\NormalTok{residuals\_df }\OtherTok{\textless{}{-}} \FunctionTok{data.frame}\NormalTok{(}
  \AttributeTok{Fitted\_Values =} \FunctionTok{fitted}\NormalTok{(model),}
  \AttributeTok{Residuals =} \FunctionTok{residuals}\NormalTok{(model)}
\NormalTok{)}

\FunctionTok{library}\NormalTok{(ggplot2)}
\FunctionTok{ggplot}\NormalTok{(residuals\_df, }\FunctionTok{aes}\NormalTok{(}\AttributeTok{x =}\NormalTok{ Fitted\_Values, }\AttributeTok{y =}\NormalTok{ Residuals)) }\SpecialCharTok{+}
  \FunctionTok{geom\_point}\NormalTok{() }\SpecialCharTok{+}
  \FunctionTok{geom\_hline}\NormalTok{(}\AttributeTok{yintercept =} \DecValTok{0}\NormalTok{, }\AttributeTok{linetype =} \StringTok{"dashed"}\NormalTok{, }\AttributeTok{color =} \StringTok{"red"}\NormalTok{) }\SpecialCharTok{+}
  \FunctionTok{labs}\NormalTok{(}\AttributeTok{title =} \StringTok{"Residuals vs Fitted Values with Observed Pattern"}\NormalTok{,}
       \AttributeTok{x =} \StringTok{"Fitted Values"}\NormalTok{,}
       \AttributeTok{y =} \StringTok{"Residuals"}\NormalTok{)}
\end{Highlighting}
\end{Shaded}

\includegraphics{PS12-Dilip_files/figure-latex/unnamed-chunk-16-1.pdf}
Normality of errors :

lets feed the residual to a normal QQ plot:

\begin{Shaded}
\begin{Highlighting}[]
\FunctionTok{ggplot}\NormalTok{(aug\_model\_df, }\FunctionTok{aes}\NormalTok{(}\AttributeTok{sample =}\NormalTok{ .resid)) }\SpecialCharTok{+} \FunctionTok{stat\_qq}\NormalTok{() }\SpecialCharTok{+} \FunctionTok{stat\_qq\_line}\NormalTok{()}
\end{Highlighting}
\end{Shaded}

\includegraphics{PS12-Dilip_files/figure-latex/unnamed-chunk-17-1.pdf}

It definitely is not straight.At the tails it deviates significantly
from the qqline. Hence, one could conclude that the residuals are barely
normal.

\hypertarget{question-3}{%
\subsubsection{Question 3:}\label{question-3}}

\hypertarget{question-3a}{%
\paragraph{Question 3a:}\label{question-3a}}

Assumption of ANOVA

\begin{enumerate}
\def\labelenumi{\arabic{enumi}.}
\tightlist
\item
  Observations are independent: Yes, Because the rats were randomly
  divided into four groups and were put in a cage, to a certain extent
  one could say this was a randomized control experiment. So the
  observations are independent.
\end{enumerate}

2.All the populations are normal : From the given QQ plots, except at
the tails, the observations does look like they fall on the straight
line. But perhaps, if we had more sample we could have been sure of the
normality. From the 35 observations for each group we have, we can
assume that the population is normal.

\begin{enumerate}
\def\labelenumi{\arabic{enumi}.}
\setcounter{enumi}{2}
\tightlist
\item
  Homoscedasticity:
\end{enumerate}

Apart from the fruit diet sample that has a standard deviation of 16.9,
other sample's standard deviation are very close to each other. To a
certain extent, one can conclude that the stds of all four samples
(16.9, 14.6,14.2,14.1) are kinda close to each other. hence
Homoskedasticity checks out.

\hypertarget{question-3b}{%
\paragraph{Question 3b:}\label{question-3b}}

\begin{Shaded}
\begin{Highlighting}[]
\NormalTok{N }\OtherTok{\textless{}{-}} \DecValTok{140}
\NormalTok{n}\OtherTok{\textless{}{-}} \DecValTok{35}

\NormalTok{fruitMean }\OtherTok{\textless{}{-}} \FloatTok{83.5}
\NormalTok{fruitSD }\OtherTok{\textless{}{-}} \FloatTok{16.9}
\NormalTok{carbsMean }\OtherTok{\textless{}{-}} \FloatTok{92.3}
\NormalTok{carbsSD }\OtherTok{\textless{}{-}} \FloatTok{14.6}
\NormalTok{meatMean }\OtherTok{\textless{}{-}} \FloatTok{88.6}
\NormalTok{meatSD }\OtherTok{\textless{}{-}} \FloatTok{14.2}
\NormalTok{mixedMean }\OtherTok{\textless{}{-}} \FloatTok{99.4}
\NormalTok{mixedSD }\OtherTok{\textless{}{-}} \FloatTok{14.1}


\NormalTok{means }\OtherTok{\textless{}{-}} \FunctionTok{c}\NormalTok{(fruitMean,carbsMean,meatMean,mixedMean)}
\NormalTok{SD }\OtherTok{\textless{}{-}} \FunctionTok{c}\NormalTok{(fruitSD,carbsSD,meatSD,mixedSD)}
\NormalTok{grandMean }\OtherTok{\textless{}{-}} \FunctionTok{mean}\NormalTok{(means)}
\NormalTok{SSB }\OtherTok{\textless{}{-}}\NormalTok{ n }\SpecialCharTok{*}\NormalTok{ (fruitMean}\SpecialCharTok{{-}}\NormalTok{grandMean)}\SpecialCharTok{\^{}}\DecValTok{2} \SpecialCharTok{+}\NormalTok{ n }\SpecialCharTok{*}\NormalTok{ (carbsMean}\SpecialCharTok{{-}}\NormalTok{grandMean)}\SpecialCharTok{\^{}}\DecValTok{2} \SpecialCharTok{+}\NormalTok{ n }\SpecialCharTok{*}\NormalTok{ (meatMean}\SpecialCharTok{{-}}\NormalTok{grandMean)}\SpecialCharTok{\^{}}\DecValTok{2} \SpecialCharTok{+}\NormalTok{ n }\SpecialCharTok{*}\NormalTok{ (mixedMean}\SpecialCharTok{{-}}\NormalTok{grandMean)}\SpecialCharTok{\^{}}\DecValTok{2}
\NormalTok{betweenDF }\OtherTok{\textless{}{-}} \DecValTok{4{-}1}
\NormalTok{between.meansquare }\OtherTok{\textless{}{-}}\NormalTok{ SSB}\SpecialCharTok{/}\NormalTok{betweenDF}
\NormalTok{SSB}
\end{Highlighting}
\end{Shaded}

\begin{verbatim}
## [1] 4698.75
\end{verbatim}

\begin{Shaded}
\begin{Highlighting}[]
\NormalTok{betweenDF}
\end{Highlighting}
\end{Shaded}

\begin{verbatim}
## [1] 3
\end{verbatim}

\begin{Shaded}
\begin{Highlighting}[]
\NormalTok{between.meansquare}
\end{Highlighting}
\end{Shaded}

\begin{verbatim}
## [1] 1566.25
\end{verbatim}

\begin{Shaded}
\begin{Highlighting}[]
\NormalTok{ssw }\OtherTok{\textless{}{-}}\NormalTok{ (n}\DecValTok{{-}1}\NormalTok{) }\SpecialCharTok{*}\NormalTok{ fruitSD}\SpecialCharTok{\^{}}\DecValTok{2} \SpecialCharTok{+}\NormalTok{ (n}\DecValTok{{-}1}\NormalTok{) }\SpecialCharTok{*}\NormalTok{ carbsSD}\SpecialCharTok{\^{}}\DecValTok{2} \SpecialCharTok{+}\NormalTok{ (n}\DecValTok{{-}1}\NormalTok{) }\SpecialCharTok{*}\NormalTok{ meatSD}\SpecialCharTok{\^{}}\DecValTok{2} \SpecialCharTok{+}\NormalTok{ (n}\DecValTok{{-}1}\NormalTok{) }\SpecialCharTok{*}\NormalTok{ mixedSD}\SpecialCharTok{\^{}}\DecValTok{2}
\NormalTok{dfw }\OtherTok{\textless{}{-}}\NormalTok{ N}\DecValTok{{-}4} \CommentTok{\# 4 groups}
\NormalTok{within.meansqaure }\OtherTok{\textless{}{-}}\NormalTok{ ssw}\SpecialCharTok{/}\NormalTok{dfw}
\NormalTok{ssw}
\end{Highlighting}
\end{Shaded}

\begin{verbatim}
## [1] 30573.48
\end{verbatim}

\begin{Shaded}
\begin{Highlighting}[]
\NormalTok{within.meansqaure}
\end{Highlighting}
\end{Shaded}

\begin{verbatim}
## [1] 224.805
\end{verbatim}

\begin{Shaded}
\begin{Highlighting}[]
\NormalTok{dfw}
\end{Highlighting}
\end{Shaded}

\begin{verbatim}
## [1] 136
\end{verbatim}

\begin{Shaded}
\begin{Highlighting}[]
\NormalTok{F }\OtherTok{\textless{}{-}}\NormalTok{ between.meansquare}\SpecialCharTok{/}\NormalTok{within.meansqaure}
\NormalTok{F}
\end{Highlighting}
\end{Shaded}

\begin{verbatim}
## [1] 6.967149
\end{verbatim}

\begin{Shaded}
\begin{Highlighting}[]
\DecValTok{1} \SpecialCharTok{{-}} \FunctionTok{pf}\NormalTok{(F, }\AttributeTok{df1 =}\NormalTok{ betweenDF, }\AttributeTok{df2 =}\NormalTok{ dfw)}
\end{Highlighting}
\end{Shaded}

\begin{verbatim}
## [1] 0.0002140835
\end{verbatim}

Hence, SSB = 4698.75, BDF = 3, between mean square = 1566.25

SSW = 30573.48, within mean square = 224.805, WDF = 136

F = 6.967149

Total = 35272.23

Total DF = 136 + 3 = 139

p-value = 0.00021

\hypertarget{question-3c}{%
\paragraph{Question 3c:}\label{question-3c}}

There is infact strong evidence against the null hypothesis with a
p-value of 0.00021.Yes, our sample size is small. But to a certain
extent we can conclude that there is significant difference in the means
of the group which means one of the four groups has a significant
difference in the amount of weight gained. Do we know which group? we
don't. But with further analysis, one could find out the answer.

\hypertarget{question-4-empathy-vs-game}{%
\subsubsection{Question 4: Empathy vs
Game}\label{question-4-empathy-vs-game}}

\hypertarget{question-4a}{%
\paragraph{Question 4a}\label{question-4a}}

\begin{Shaded}
\begin{Highlighting}[]
\NormalTok{game\_data }\OtherTok{\textless{}{-}} \FunctionTok{read.table}\NormalTok{(}\StringTok{"GameEmpathy.txt"}\NormalTok{, }\AttributeTok{header =} \ConstantTok{TRUE}\NormalTok{)}
\FunctionTok{head}\NormalTok{(game\_data)}
\end{Highlighting}
\end{Shaded}

\begin{verbatim}
##      sex game.type identify  empathy
## 1 female   neutral 3.333333 5.285714
## 2 female   neutral 1.833333 5.571429
## 3   male   neutral 1.000000 4.714286
## 4 female   neutral 1.000000 5.571429
## 5 female   neutral 3.333333 3.142857
## 6 female   neutral 1.000000 5.571429
\end{verbatim}

Visualization:

\begin{Shaded}
\begin{Highlighting}[]
\FunctionTok{boxplot}\NormalTok{(empathy }\SpecialCharTok{\textasciitilde{}}\NormalTok{ game.type, }\AttributeTok{data =}\NormalTok{ game\_data, }\AttributeTok{col =} \StringTok{"red"}\NormalTok{, }\AttributeTok{main =} \StringTok{"Empathy for different types of Games"}\NormalTok{)}
\end{Highlighting}
\end{Shaded}

\includegraphics{PS12-Dilip_files/figure-latex/unnamed-chunk-23-1.pdf}

\begin{Shaded}
\begin{Highlighting}[]
\NormalTok{GTA }\OtherTok{\textless{}{-}} \FunctionTok{subset}\NormalTok{(game\_data, game.type }\SpecialCharTok{==} \StringTok{"GTA"}\NormalTok{)}
\NormalTok{HalfLife }\OtherTok{\textless{}{-}} \FunctionTok{subset}\NormalTok{(game\_data, game.type }\SpecialCharTok{==} \StringTok{"HalfLife"}\NormalTok{)}
\NormalTok{neutral }\OtherTok{\textless{}{-}} \FunctionTok{subset}\NormalTok{(game\_data, game.type }\SpecialCharTok{==} \StringTok{"neutral"}\NormalTok{)}

\CommentTok{\#sample empthy observations for each games}
\FunctionTok{mean}\NormalTok{(GTA}\SpecialCharTok{$}\NormalTok{empathy)}
\end{Highlighting}
\end{Shaded}

\begin{verbatim}
## [1] 5.029762
\end{verbatim}

\begin{Shaded}
\begin{Highlighting}[]
\FunctionTok{mean}\NormalTok{(HalfLife}\SpecialCharTok{$}\NormalTok{empathy)}
\end{Highlighting}
\end{Shaded}

\begin{verbatim}
## [1] 5.293939
\end{verbatim}

\begin{Shaded}
\begin{Highlighting}[]
\FunctionTok{mean}\NormalTok{(neutral}\SpecialCharTok{$}\NormalTok{empathy)}
\end{Highlighting}
\end{Shaded}

\begin{verbatim}
## [1] 5.05381
\end{verbatim}

\begin{Shaded}
\begin{Highlighting}[]
\FunctionTok{NROW}\NormalTok{(GTA)}
\end{Highlighting}
\end{Shaded}

\begin{verbatim}
## [1] 48
\end{verbatim}

\begin{Shaded}
\begin{Highlighting}[]
\FunctionTok{NROW}\NormalTok{(HalfLife)}
\end{Highlighting}
\end{Shaded}

\begin{verbatim}
## [1] 55
\end{verbatim}

\begin{Shaded}
\begin{Highlighting}[]
\FunctionTok{NROW}\NormalTok{(neutral)}
\end{Highlighting}
\end{Shaded}

\begin{verbatim}
## [1] 50
\end{verbatim}

ANOVA test:

\begin{Shaded}
\begin{Highlighting}[]
\NormalTok{anova\_model }\OtherTok{\textless{}{-}} \FunctionTok{aov}\NormalTok{(empathy }\SpecialCharTok{\textasciitilde{}}\NormalTok{ game.type, }\AttributeTok{data =}\NormalTok{ game\_data)}
\FunctionTok{summary}\NormalTok{(anova\_model)}
\end{Highlighting}
\end{Shaded}

\begin{verbatim}
##              Df Sum Sq Mean Sq F value Pr(>F)
## game.type     2   2.25   1.125   1.092  0.338
## Residuals   150 154.47   1.030
\end{verbatim}

\begin{Shaded}
\begin{Highlighting}[]
\NormalTok{F }\OtherTok{\textless{}{-}} \FloatTok{1.092}
\NormalTok{bdf }\OtherTok{\textless{}{-}} \DecValTok{2}
\NormalTok{wdf }\OtherTok{\textless{}{-}} \DecValTok{150}
\DecValTok{1} \SpecialCharTok{{-}} \FunctionTok{pf}\NormalTok{(F, }\AttributeTok{df1 =}\NormalTok{ bdf, }\AttributeTok{df2 =}\NormalTok{ wdf)}
\end{Highlighting}
\end{Shaded}

\begin{verbatim}
## [1] 0.338197
\end{verbatim}

Because we dont have a tiny p-value (0.05), we can conclude that indeed
the null hypotheses is true, there is no significant difference in the
means between the three samples.

\hypertarget{question-4b}{%
\paragraph{Question 4b:}\label{question-4b}}

To understand the relationship between identification and empathy among
gamers who played different games, lets find the correlation and
visualize scatterplots.

\begin{enumerate}
\def\labelenumi{\roman{enumi}.}
\tightlist
\item
  Student who played neutral games:
\end{enumerate}

\begin{Shaded}
\begin{Highlighting}[]
\FunctionTok{plot}\NormalTok{(neutral}\SpecialCharTok{$}\NormalTok{identify, neutral}\SpecialCharTok{$}\NormalTok{empathy,}
     \AttributeTok{xlab =} \StringTok{"Identification"}\NormalTok{, }\AttributeTok{ylab =} \StringTok{"Empathy"}\NormalTok{,}
     \AttributeTok{main =} \StringTok{"Identification vs Empathy for (Neutral Gamers)"}\NormalTok{)}
\end{Highlighting}
\end{Shaded}

\includegraphics{PS12-Dilip_files/figure-latex/unnamed-chunk-28-1.pdf}
Correlation :

\begin{Shaded}
\begin{Highlighting}[]
\CommentTok{\# Calculate correlation coefficient}
\NormalTok{corr\_neutral }\OtherTok{\textless{}{-}} \FunctionTok{cor}\NormalTok{(neutral}\SpecialCharTok{$}\NormalTok{identify, neutral}\SpecialCharTok{$}\NormalTok{empathy)}
\NormalTok{corr\_neutral}
\end{Highlighting}
\end{Shaded}

\begin{verbatim}
## [1] 0.08991878
\end{verbatim}

\begin{Shaded}
\begin{Highlighting}[]
\CommentTok{\#calculate the p{-}value}
\NormalTok{cor\_test\_neutral }\OtherTok{\textless{}{-}} \FunctionTok{cor.test}\NormalTok{(neutral}\SpecialCharTok{$}\NormalTok{identify, neutral}\SpecialCharTok{$}\NormalTok{empathy, }\AttributeTok{method =} \StringTok{"pearson"}\NormalTok{)}
\NormalTok{p\_value\_neutral }\OtherTok{\textless{}{-}}\NormalTok{ cor\_test\_neutral}\SpecialCharTok{$}\NormalTok{p.value}
\NormalTok{p\_value\_neutral}
\end{Highlighting}
\end{Shaded}

\begin{verbatim}
## [1] 0.5345996
\end{verbatim}

A small positive correlation between identification and empathy of 0.089
is observed in neutral players

\begin{enumerate}
\def\labelenumi{\roman{enumi}.}
\setcounter{enumi}{1}
\tightlist
\item
  Half Life:
\end{enumerate}

\begin{Shaded}
\begin{Highlighting}[]
\FunctionTok{plot}\NormalTok{(HalfLife}\SpecialCharTok{$}\NormalTok{identify, HalfLife}\SpecialCharTok{$}\NormalTok{empathy,}
     \AttributeTok{xlab =} \StringTok{"Identification"}\NormalTok{, }\AttributeTok{ylab =} \StringTok{"Empathy"}\NormalTok{,}
     \AttributeTok{main =} \StringTok{"Identification vs Empathy (HalfLife)"}\NormalTok{)}
\end{Highlighting}
\end{Shaded}

\includegraphics{PS12-Dilip_files/figure-latex/unnamed-chunk-30-1.pdf}

\begin{Shaded}
\begin{Highlighting}[]
\NormalTok{corr\_HalfLife}\OtherTok{\textless{}{-}} \FunctionTok{cor}\NormalTok{(HalfLife}\SpecialCharTok{$}\NormalTok{identify, HalfLife}\SpecialCharTok{$}\NormalTok{empathy, }\AttributeTok{method =} \StringTok{"pearson"}\NormalTok{)}
\NormalTok{corr\_HalfLife}
\end{Highlighting}
\end{Shaded}

\begin{verbatim}
## [1] 0.07164441
\end{verbatim}

\begin{Shaded}
\begin{Highlighting}[]
\CommentTok{\#calculate the p{-}value}
\NormalTok{cor\_test\_HalfLife }\OtherTok{\textless{}{-}} \FunctionTok{cor.test}\NormalTok{(HalfLife}\SpecialCharTok{$}\NormalTok{identify, HalfLife}\SpecialCharTok{$}\NormalTok{empathy, }\AttributeTok{method =} \StringTok{"pearson"}\NormalTok{)}
\NormalTok{p\_value\_HalfLife }\OtherTok{\textless{}{-}}\NormalTok{ cor\_test\_HalfLife}\SpecialCharTok{$}\NormalTok{p.value}
\NormalTok{p\_value\_HalfLife}
\end{Highlighting}
\end{Shaded}

\begin{verbatim}
## [1] 0.6032072
\end{verbatim}

\begin{enumerate}
\def\labelenumi{\roman{enumi}.}
\setcounter{enumi}{2}
\tightlist
\item
  GTA:
\end{enumerate}

\begin{Shaded}
\begin{Highlighting}[]
\FunctionTok{plot}\NormalTok{(GTA}\SpecialCharTok{$}\NormalTok{identify, GTA}\SpecialCharTok{$}\NormalTok{empathy,}
     \AttributeTok{xlab =} \StringTok{"Identification"}\NormalTok{, }\AttributeTok{ylab =} \StringTok{"Empathy"}\NormalTok{,}
     \AttributeTok{main =} \StringTok{"Identification vs Empathy (GTA)"}\NormalTok{)}
\end{Highlighting}
\end{Shaded}

\includegraphics{PS12-Dilip_files/figure-latex/unnamed-chunk-32-1.pdf}
Correlation:

\begin{Shaded}
\begin{Highlighting}[]
\CommentTok{\# Calculate correlation coefficient}
\NormalTok{corr\_GTA }\OtherTok{\textless{}{-}} \FunctionTok{cor}\NormalTok{(GTA}\SpecialCharTok{$}\NormalTok{identify, GTA}\SpecialCharTok{$}\NormalTok{empathy)}
\NormalTok{corr\_GTA}
\end{Highlighting}
\end{Shaded}

\begin{verbatim}
## [1] -0.2722745
\end{verbatim}

\begin{Shaded}
\begin{Highlighting}[]
\CommentTok{\#calculate the p{-}value}
\NormalTok{cor\_test\_GTA }\OtherTok{\textless{}{-}} \FunctionTok{cor.test}\NormalTok{(GTA}\SpecialCharTok{$}\NormalTok{identify, GTA}\SpecialCharTok{$}\NormalTok{empathy, }\AttributeTok{method =} \StringTok{"pearson"}\NormalTok{)}
\NormalTok{p\_value\_GTA }\OtherTok{\textless{}{-}}\NormalTok{ cor\_test\_GTA}\SpecialCharTok{$}\NormalTok{p.value}
\NormalTok{p\_value\_GTA}
\end{Highlighting}
\end{Shaded}

\begin{verbatim}
## [1] 0.06118009
\end{verbatim}

Adjusting the p-values:

\begin{Shaded}
\begin{Highlighting}[]
\CommentTok{\#  adjust for multiple testing using bonferroni method}
\NormalTok{p\_values }\OtherTok{\textless{}{-}} \FunctionTok{c}\NormalTok{(p\_value\_neutral, p\_value\_HalfLife, p\_value\_GTA)}
\NormalTok{p\_value\_adjust }\OtherTok{\textless{}{-}} \FunctionTok{p.adjust}\NormalTok{(p\_values, }\AttributeTok{method =} \StringTok{"bonferroni"}\NormalTok{)}
\NormalTok{p\_value\_adjust}
\end{Highlighting}
\end{Shaded}

\begin{verbatim}
## [1] 1.0000000 1.0000000 0.1835403
\end{verbatim}

Well, with the adjusted p-values using bonferroni method, we can clearly
see that these are not tiny. With the assumed significance level of
0.05/3, we can say the data is consistent with the null hypotheses that
there is no significant difference between identity and empathy among
students that play gta, halflife or neutral games. Are we 100\% sure?
The sample size we have is about 50 for each class which in my opinion
is too less to draw conclusion on the population. But perhaps, with more
data we could be 100\% sure.

\end{document}
